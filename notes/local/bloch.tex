\documentclass{article}
\usepackage[utf8]{inputenc}
\usepackage{amsmath}

\title{Notes}
\author{Roger Luo}
\date{September 2017}

\newcommand{\mean}[1]{\langle #1 \rangle}
\newcommand{\rket}[1]{|#1\rangle}
\newcommand{\lket}[1]{\langle #1 |}
\newcommand{\norm}[1]{\left\lVert#1\right\rVert}
\newcommand{\conj}[1]{\overline{#1}}
\newcommand{\diff}[1]{\partial_{\conj{\theta}} #1}

\begin{document}

\maketitle

\section{Sampling configs from arbitrary quantum state on given basis}\label{sampling}

A pair of basis for a single qubit could be

\begin{equation}
\begin{aligned}
\rket{\epsilon_0} &= \cos{\theta}\rket{0} + \sin{\theta} e^{i\phi}\rket{1}\\
\rket{\epsilon_1} &= -\sin{\theta} e^{-i\phi}\rket{0} + \cos{\theta}\rket{1}
\end{aligned}
\end{equation}

and their eigen value should be $\{-1, 1\}$, thus

\begin{equation}
P
\begin{pmatrix}
1 & 0\\
0 & -1
\end{pmatrix}
P^{-1}
\end{equation}

and P's colums are the its eigenvectors

\begin{equation}
P =
\begin{pmatrix}
\cos{\theta} & -\sin{\theta} e^{-i\phi}\\
\sin{\theta} e^{i\phi} & \cos{\theta}
\end{pmatrix}
\end{equation}

and $P^{-1}$ is

\begin{equation}
P^{-1} =
\begin{pmatrix}
\cos{\theta} & \sin{\theta}e^{-i\phi}\\
-\sin{\theta}e^{i\phi} & \cos{\theta}
\end{pmatrix}
\end{equation}

Therefore, the operator is

\begin{equation}
\begin{pmatrix}
\cos{2\theta} & \sin{2\theta}e^{-i\phi}\\
\sin{2\theta}e^{i\phi} & -\cos{2\theta}
\end{pmatrix}
\end{equation}

\section{Summary}

Operator on bloch sphere

\begin{equation}
O = \begin{pmatrix}
\cos{2\theta} & \sin{2\theta}e^{-i\phi}\\
\sin{2\theta}e^{i\phi} & -\cos{2\theta}
\end{pmatrix}
\end{equation}

its eigen vectors and corresponding eigen values are

\begin{equation}
\begin{aligned}
\begin{pmatrix}
\cos{\theta}\\
\sin{\theta}e^{i\phi}
\end{pmatrix}&, 1\\
\begin{pmatrix}
-\sin{\theta}e^{-i\phi}\\
\cos{\theta}
\end{pmatrix}&, -1\\
\end{aligned}
\end{equation}

To transform a state vector $\rket{\Psi}$ on $\sigma_z$ basis to $O$ basis by matrix $P$ in section \ref{sampling}:

\begin{equation}
\begin{aligned}
\rket{\Psi} &= A_0\rket{\epsilon_0} + A_1\rket{\epsilon_1} \\
            &= A_0 P\rket{0} + A_1 P \rket{1}
\end{aligned}
\end{equation}

or on $\sigma_z$ basis

\begin{equation}
P \begin{pmatrix}
A_0\\
A_1
\end{pmatrix}
\end{equation}

the inverse tranformation is:

given a state $\rket{\Psi}$ on $\sigma_z$ basis, we have

\begin{equation}
\begin{aligned}
\rket{\Psi} &= A_0\rket{0} + A_1\rket{1}\\
            &= A_0 P^{-1}\rket{\epsilon_0} + A_1 P^{-1} \rket{\epsilon_1}
\end{aligned}
\end{equation}

\section{Many-body case}

$\sigma_z$ basis: $\rket{s_0}, \rket{s_1}$
$O^k$ basis: $\rket{m_0}, \rket{m_1}$

Give a quantum state $\rket{\Psi}$ on $n$ particles on $\sigma_z$'s basis

\begin{equation}
\begin{aligned}
\rket{\Psi} &= \sum_{s_i\in\{0, 1\}} A_{s_1,\cdots, s_n}\bigotimes_{i=1}^{n}\rket{s_i}\\
            &= \sum_{m_i\in\{0, 1\}} A_{m_1, \cdots, m_n}\bigotimes_{i=1}^{n} P_i^{-1}\rket{m_i}\\
            &= \bigotimes_{i=1}^{n} P_i^{-1} \sum_{m_i\in\{0, 1\}}A_{m_1, \cdots, m_n} \bigotimes_{i=1}^{n} \rket{m_i}
\end{aligned}
\end{equation}

\end{document}
